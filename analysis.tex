\chapter{Analysis of~Our Solution}
\label{kap:analysis}

%In~this chapter, we discuss how our solution~meets requirements established in~chapter \ref{kap:definition}.
In~Chapter \ref{kap:definition}, we established several requirements concerning \emph{usability}, \emph{security} and \emph{accuracy} of~an~electronic election~system. In~this chapter we briefly and informally analyse how our solution~presented in~Chapters \ref{kap:solution} and \ref{kap:implementation} meets these requirements.

\section{Usability}
Here, we discuss the~usability of~our electronic voting system concerning these points:
\begin{itemize}
\item \textbf{Easy-to-use user interface.} Our user interface is provided by an~online application. Users can~easily log in~using the~usual method used by our faculty. The~actual voting is done by clicking on~a~few radio buttons and clicking on~two submit buttons. The~voter is given the~information~on~the~number of~candidates they can~vote in~favour of. On~the~other hand, we could say that~giving answer for~every candidate might be time consuming. But, the~\emph{neutral} option~is set as the~default option. It means that~the~voting time can~be reduced just to~giving \emph{yes} answers to~the~desired candidates while the~others are left with the~default answer.
\item \textbf{Mobility.} The~voter is not restricted to~a~particular type of~device, nor have they to~stick on~an~individual device. What~is more, no voting application~is needed to~be installed. The~voters can~access voting everywhere. The~only thing they need is a~device with a~supported Internet browser and an~Internet connection.
\item \textbf{Cost-effectiveness.} The~system requires two to~three computer devices on the~server side, which can~be costly depending on~the~resources of~the~faculty. Yet, all the~software used for~the~system is open-source, so that~it does not require any additional financial resources.
\item \textbf{Confirmation.} When the~vote is formed, the~voter can~still change their mind and create another vote before~the~former would be sent. Moreover, using the~electronic voting, the~voter can~cast a~vote as many times as they want to~and only the~most current one remains recorded. Yet, confirmation~is not fully implemented in~our solution. For~example, the~Estonian~voting protocol, described in~Chapter \ref{kap:existing}, implements an~application~that~the~voter can~use to~check their vote. It might be possible to~implement such an~application~for~our electronic election~system, too. However, this is not included in~our implementation~partly due to~preserving \emph{incoercibility}.
\end{itemize}
\section{Security}
In~this section, we discuss how secure our solution~is. We analyse these requirements:
\begin{itemize}
\item \textbf{Secrecy and anonymity.} When the~vote is formed and sent to~the~server for~vote collection, it is in~an~encrypted state and the~server does not possess~the~election~private key, which must be used to~decrypt the~votes. Instead, this private key is shared among members of~the~voting commission, using a~secret-sharing scheme. When the~votes are transferred to~the~machine for~vote counting, they are no more associated with voters' identities. Therefore, they can~be decrypted and counted there. Keeping secrecy and anonymity relies mostly on~the~election~commission. They should not provide the~server for~vote collection~with the~election~private key. They must not provide the~machine for~vote counting with voters' identities associated with the~votes, either. The~most vulnerable part of~our code is retrieving the~encrypted votes from~the~database of~votes and saving them to~the~USB flash drive. At~this very moment, a~bug or undesired intervention~can~cause lost of~secrecy and anonymity.
\item \textbf{Reliability.} Our voting scheme is very minimal and mostly uses known techniques. Votes are stored in~a~database and associated with an~individual identity for~most of~the~time. No anonymous channels, described in~Chapter \ref{kap:existing}, are used. Therefore, chances that~the~votes are dropped or substituted while they are transferred or stored are low. Moreover, a~suspicious voter can~cast their vote again~if they think there is chance that~is has not been recorded correctly. Reliability also rests on~the~election~commission~during administration~of~paper voting and during the~maintenance of~server-side devices used. On~the~other hand, voters are responsible for~they devices and they can~contain~malware, which can~affect the~election. Our implementation~relies on~at~least two cryptographic libraries. It means that~there can~be some bugs in~them, as well. If we find that~out, we are ready to~update our implementation~according to~the~findings. 
\item \textbf{Incoercibility.} The~voter can~change their vote in~any moment. That~means that, from~the~client side, there is no such a~way in~which the~voter can~prove how they voted most recently. On~the~other hand, the~encrypted votes are stored together with \emph{UK login}s of~the~voters' and the~voters have access to~the~encrypted vote stored in~their Internet browser during the~voting session. %If an~attacker succeeds in~accessing the~database of~votes, they can~use comparable methods to~check how an~individual voted when they have access to~the~encrypted vote from~the~individual's Internet browser.
If an~attacker succeeds in~accessing the~database of~votes, they can~compare the~encrypted voter's vote in~the~database with the~encrypted vote from~the~voter's Internet browser.
\end{itemize}
\section{Accuracy}
It is important to~make sure that~only valid votes are recorded, all the~votes are accurately counted and the~right result is displayed. Now, we discuss how our solution~meets our accuracy requirements:
\begin{itemize}
\item \textbf{Authorisation.} In~our system, voters are authorised twice. First, they need to~log in~using \emph{Cosing}, an~authentication~system used by our University, before~the~voting application~is even displayed to~them. When they send their vote together with their identity provided by \emph{Cosign}, the~server for~vote collection~checks whether they are allowed to~vote. If not, the~sent vote is, simply, not recorded. %This means that~only students of~our University have access to~the~voting application.
\item \textbf{Uniqueness and limitation.} Every voter participating in~the~electronic election~is allowed to~cast as many votes as they want to. However, only the~most recent vote remains recorded. As soon~as they participate in~the~paper election, their electronic votes are erased and they are not able to~participate in~the~election~anymore. The~voting application~checks whether the~voter does not vote for~more than~allowed number of~candidates in~a~single vote. Due to~the~fact that~the~code of~the~voting application~can~be changed in~the~Internet browser and an~invalid vote can~be sent, this is double-checked during the~counting phase of~our election~system.
\item \textbf{Persistence.} Sent votes are stored in~the~database and handled with according to~the~voting scheme. They are transferred only once—from~the~server for~vote collection~to~the~machine for~vote counting. However, the~votes can~be irreversibly changed by the~election~commission~if they accidentally make a~record about a~voter that~has not been involved in~paper voting, yet.
\item \textbf{Correct computation.} The~machine for~vote counting is responsible for~the~correct computation. Therefore, this depends on~the~correctness of~the~particular implementation~of~the~machine for~vote counting. We have not discovered any bugs regarding vote computation~in~our implementation.
\end{itemize}