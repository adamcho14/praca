\chapter{Implementation of~Our Solution}
\label{kap:implementation}
Our task for this thesis is to develop a functioning electronic election system that can be used by our faculty's academic senate. In this chapter, we describe some important implementation details and we provide examples of important parts of our code.

\section{Overview}
For the desired system, we implement the electronic voting scheme described in chapter \ref{kap:solution}. This scheme consists of several entities, which together make our election system and some of which are represented by computers. Such \emph{players} have to be implemented independently, each running its own program that provides all its functionality and communication with other entities. Analysis of the final solution can be found in chapter \ref{kap:analysis}.

We have chosen \emph{Python} and \emph{JavaScript} as our primary programming languages. For databases, we use \emph{SQL}, a standard language in database programming.

In order to run the system correctly, there have to be some technical requirements accomplished. On the client-side, the voter needs to have one of these supported Internet browsers: \emph{Mozilla Firefox}, \emph{Google Chrome} or \emph{Safari}. The system is developed to run on \emph{Linux} distributions and the system can be build from existing Python files. In order to provide the authentication, the server needs to have $Apache 2$ server installed with a module that provide CoSign functionality. More can be read on webpages of our faculty.
\section{Database}
\section{Voting Application}
\section{Server for Vote Collection}
\section{Machine for Vote Counting}
\section{Supplementary Programs}