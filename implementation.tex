\chapter{Implementation of~Our Solution}
\label{kap:implementation}
Our task for this thesis is to~develop a~functioning electronic election system that~can~be used by~our faculty's academic senate. In~this chapter, we describe some important implementation details and we provide examples of~important parts of~our code.

\section{Overview}
For the~desired system, we implement the~electronic voting scheme described in~Chapter \ref{kap:solution}. This scheme consists of~several entities—players, which together make our election system and some of~which are represented by~computers. These players have to~be implemented independently, each running its own program that~provides all its functionality and communication with other entities. Analysis of~the~final solution can~be found in~Chapter \ref{kap:analysis}.

We have chosen \emph{Python} \cite{Python} and \emph{JavaScript} \cite{JavaScript} as our primary programming languages. For databases, we use \emph{SQL} \cite{SQL}, a~standard language in~database programming. Our used encoding is \emph{UTF-8}.
\bigbreak
We name this system \textbf{SEVAS} (\textbf System of~\textbf Electronic \textbf Voting for \textbf Academic \textbf Senate). The~source code for this system is attached to~this thesis and consists of~several Python and JavaScript files that~provide the~voting functionality and the~administration of~votes, and of~some additional configuration, CSS and database files. More information about~the~source files can~be found in~the~included \texttt{README.md} file. The code for~this system is publicly available and can be found on~GitHub. The reader can access it here: \url{github.com/adamcho14/SEVAS.git}.

In~this election system, we have three possible answers for~each candidate: \emph{positive}, \emph{negative} and \emph{neutral}. However, if one wishes, the code for~this system can be easily changed to~use a~different number of~possible answers.

\subsection{Technical Requirements}
In~order to~run the~system correctly, there have to~be some technical requirements accomplished. On the~client-side, the~voter needs to~have one of~these supported Internet browsers: \emph{Mozilla~Firefox}, \emph{Google Chrome} or \emph{Safari}.

The~system is developed to~run on \emph{Linux} distributions and the~system can~be built from~existing \emph{Python} source files. In~order to~provide the~authentication, the~\emph{Linux} voting server needs to~have \emph{Apache 2} server installed with a~module that~provides \emph{Cosign} functionality. %More about~configuration of~\emph{Cosign} can~be read on our faculty webpages (cite).
More about~\emph{Cosign} can~be read in~Chapter \ref{kap:solution}.

The~voting is provided by~a~series of~\emph{CGI} scripts. Common Gateway Interface (CGI) is a~protocol used to~run external programs—\emph{CGI} scripts—under a~server. It forwards the~request sent by~the~client to~the~external program, which is executed, and the~output of~the~program in~the~form of~a~response is sent via~the~server to~the~client \cite{CGI}. These scripts are very often used to~generate dynamic web content.
%The~way in~which the~scripts are executed is determined by~the~server.

In~order to~manage the~keys and certificates, \emph{OpenSSL} has to~be installed on the~machine for vote counting. It is a~full-featured toolkit enabling functionality of~\emph{SSL} and \emph{TLS} protocols, and also a~cryptography library \cite{OpenSSL}.

\subsection{External Libraries and Packages Used} 
In~almost every file, we include packages \texttt{cgi} and \texttt{cgitb} to~provide the~\emph{CGI} functionality. We also use several external libraries or packages to~implement services that~we do not program directly. They are used in~different parts of~our election system:
\begin{itemize}
\item \textbf{Front end JavaScript vote forming and vote encryption.} JavaScript cryptography library \emph{PKI.js} (cite)
\item \textbf{Back end Python internet voting.} We import Python packages \texttt{os}, \texttt{rsa}, \texttt{sqlite3} and \texttt{base64}
\item \textbf{Python vote collection.} Packages imported are \texttt{rsa}, \texttt{sqlite3} and \texttt{base64}.
\item \textbf{Machine for vote counting code.} In~this code, packages \texttt{json}, \texttt{sqlite3} and \texttt{M2Crypto}.
\item \textbf{Code for key and vote administration.} Here, we import packages \texttt{sqlite}, \texttt{rsa}, \texttt{M2Crypto}, \texttt{secretsharing} and \texttt{base64}.
\end{itemize}

\section{Database}
We use an~open-source database library \emph{SQLite} \cite{SQLite}, which is an~embedded database with no separate server process. We strongly believe that~this database library is sufficient for our system because it is used as a~server-side data~storage. The~client does not communicate directly with the~database, but they either send data~to~the~server, which are stored in~the~database by~the~server, or all Select statements are first posted by~the~server and their results are sent to~the~client. We, too, do not expect very high traffic because our system is created for elections with no more than~about~a~few thousand voters (our faculty has about~a~thousand students). 

We have also chosen \emph{SQLite} mostly because we find it sufficient for the~purpose of~this particular implementation, that~is to~show how our electronic election system works. If the~user of~this system wishes, \emph{MySQL}, \emph{PostgreSQL} or another \emph{SQL} server can~be chosen for the~practical use instead of~\emph{SQLite} library. It is obvious that~it requires some minor changes in~implementation, which can~be easily executed. 

Database structure of~our voting system consists of~two separate databases stored in~the~memory of~the~server for vote collection: \emph{database of~persons} and \emph{database of~votes}. There are discusses in~the~following subsections.

\subsection{Database of~Persons}
\label{sub:persons}
The~purpose of~this database is to~store information about~both the~candidates and the~voters. They are used to~display the~list of~candidates and to~provide the~list of~authorised voters. Thus, this database has two tables, whose structure is described below. Not only do we provide the~reader with the~information about~the~columns in~these particular tables, but we also show them how we create the~tables. In~this description, we use standard \emph{SQL} syntax to~present the~way in~which the~tables can~be created:  
\break
\begin{lstlisting}[language=SQL]
CREATE TABLE Candidates (
	CandID int NOT NULL PRIMARY KEY,
	LastName varchar(255) NOT NULL, 
	FirstName varchar(255) NOT NULL);	

CREATE TABLE Voters (
	UKLogin varchar(255) NOT NULL PRIMARY KEY);
\end{lstlisting}
\bigbreak
Here can~be seen that~the~table of~candidates consists of~all the~information that~is displayed to~the~voter. We suggest that~the~\texttt{CandID}s be in~ascending order starting from~number $1$. The~table of~voters contains only the~voter's \emph{UK login}. The~other information are not important for the~server and can~be fetched from~the~the~data~provided by~\emph{Cosign}. 

The~reader can~obtain~some more information about~the~usage of~these databases in~the~next sections.

\subsection{Database of~Votes}
\label{sub:votes}
This database is the~internal database of~the~server for vote collection. Its purpose is to~store encrypted votes paired with the~voters' \emph{UK logins}. The~database is made up of~a~table that~can~be created with the~following \emph{SQL} statement: \break %4095

\begin{lstlisting}[language=SQL]
CREATE TABLE Votes (
	UKLogin varchar(255) NOT NULL PRIMARY KEY,
	Vote varchar(255) NOT NULL);
\end{lstlisting}
\bigbreak

We discuss the~size of~the~column needed to~enclose the~usual vote and we suggest that~it should be adjusted according to~the~number of~candidates. Let us assume that~the~usual maximum number of~candidates in~an~election $n$ equals $10$ and we have three options for each, represented by~a~single digit. Then, the~size of~a~single vote $v$ is computed as $10\cdot 5 + 1 = 51$ bytes if the~candidates are numbered from~$1$ to~$10$ and every symbol in~the~vote is encoded by~a~single byte. \emph{S/MIME} encryption may increase the size of the message by 300\%. The~size of~\emph{base64}-encoded binary data~is 137\% of~the~size of~the~original data~\cite{SMIME, CiscoSize}. According to~our computations, a~single encoded vote ought not to~be larger than~$250$ bytes. %Yet, we have chosen a~larger size of~the~column representing the~encrypted vote.
\section{Voting}
This application provides the~voter with interface that~allows them to~authenticate and cast their vote. This application runs on the~server for vote collection as a~series \emph{CGI} scripts. The~application works as following:
\begin{enumerate}
\item When the~voter types the~\emph{WWW} address of~the~Internet elections, the~main~page with the~login~button is displayed. 
\item When the~voter clicks on the~login~button, they are redirected to~the~University's central login~webpage, where they can~authenticate.
\item When the~voter has been successfully authenticated, a~form with the~voting options, the~maximum number of~chosen options and a~button with the~label \emph{Create a~vote} is displayed. 
\item When the~voter's desired options are chosen, the~voter clicks on the~button. If they have chosen more than~the~maximum number of~options, an~alert is displayed. Otherwise, the~vote is created, encrypted and displayed on the~page together with a~button whose label says \emph{Send the~vote}.
\item When this button has been clicked on, the~voter is informed on the~page whether or not their vote has been successfully processed. At~this moment, they have successfully casted a~vote and they can~log out, or, for some of~the~reasons described in~Chapter \ref{kap:solution}, they were not allowed to~cast a~vote.
\end{enumerate}

In~order to~provide all the~functionality, these source files obtained from~the~\emph{PKI.js} library have to~be included in~the~source code for the~voting application: \break \texttt{addressparser.js}, \texttt{mimeparser-tzabbr.js}, \texttt{mimeparser.js}, \break \texttt{emailjs-mime-codec.js}, \texttt{emailjs-mime-types.js}, \texttt{emailjs-addressparser.js}, \break \texttt{punycode.js}, \texttt{emailjs-mime-builder.js}, \texttt{SMIMEEncryptionExample.js}.

\subsection{The~Form}
The~main~part of~the~voting application is the~\emph{HTML} form. It is sent using the~\emph{POST} method. It contains a~list of~candidates from~the~database of~persons, described in~Subsection \ref{sub:persons}. There are three radio buttons for every candidate, each representing one of~the~possible answers: \emph{positive}, \emph{negative}, or \emph{neutral}. To~each of~these answers a~particular integer value, obligatory for the~whole election, has been assigned. In~our implementation, positive answer is represented by~\texttt{1}, negative by~\texttt{2} and neutral by~\texttt{3}.
%The~vote is sent in~a~\emph{HTML} form using the~\emph{POST} method. 
This form also includes field \emph{vote}. This is to~be filled with the~encrypted vote. More about~vote formation can~be read in~Subsection \ref{sub:formation}.

Some parts of~our code use \emph{JavaScript} and the~voter's \emph{UK login} is displayed in~the~\emph{HTML} source code. This means that the~voter  can change the~code of~the~page and send corrupted data, such as a~vote that is not in~the~right format. Therefore, votes are validated during the~counting period.
%This means that~if the~voter knows somebody else's \emph{UK login}, they can~possibly replace their original login~with it and send their vote as somebody else's vote. Therefore, we use \emph{RSA} encryption to~encrypt the~\emph{UK login} before it is displayed to~the~user. It is necessary to~point out that~both the~encryption and decryption keys cannot be public and have to~be known only to~the~server for vote collection.

\subsection{Vote Formation and Encryption}
\label{sub:formation}
We use a~\emph{JavaScript} program to~form the~final vote from~the~voter's answers. The~format~of~the~vote is described in~Section \ref{sek:format}. Here, we describe what~is done with the~data~from~the~form in~order to~become the~encrypted vote.

It is checked whether the~number of~positive answers in~the~form does not exceed their maximum number defined by~the~election commission. All the~candidate \emph{ID}s and the~values corresponding the~checked answers are put together to~form the~vote. The~vote is, then, encrypted using \emph{S/MIME} encryption provided by~the~\emph{PKI.js} library and encoded using \emph{Base-64} encoding. For the~encryption, the~election \emph{X.509} certificate is used. More about~managing the~election keys and certificates is written in~Section \ref{sek:administration}.%We have included all the~files that~come with this library and that~we use to~encrypt the~vote in~the~attachment to~this thesis.

\begin{figure}[h]
\begin{lstlisting}[language=Java]
function createVote(radios) {
    var vote = "";
    for (i = 0; i < radios.length; i++) {
        if (radios[i].checked) {
            vote += "<";
            vote += radios[i].name.toString();
            vote += "#";
            vote += radios[i].value.toString();
            vote += ">";
        }
    }
    return vote;
    }
\end{lstlisting}
\caption{Code for the~vote formatting function}
\end{figure}
\section{Vote Collection}
Using the~\emph{POST} method, the~final vote is sent together with the~encrypted \emph{UK login} to~the~server for vote collection. There, a~\emph{CGI} script that~verifies the~voter and inserts the~vote to~the~database of~votes, described in~Subsection \ref{sub:votes}, is executed. 

The~code of~this past consists mostly of~Select, Insert and Update \emph{SQL} statements. The~script decrypts the~\emph{UK login} and checks whether the~particular voter has their record in~the~database of~persons. If this is true, then it checks whether the~voter has their vote recorded in~the~database of~votes and whether this vote has the~special \texttt{0} value. If the~value is \texttt{0}, the~vote is not recorded. If there already has a~vote been recorded with the~voter's \emph{UK login} in~the~database, the~value is updated. Otherwise, a~new record with the~voter's \emph{UK login} and the~received vote is created. Finally, it shows the~voter a~message saying whether the~vote was successfully recorded or the~voter is not allowed to~vote. 

\begin{figure}[h]
\begin{lstlisting}[language=Python]
import sqlite3

connection = sqlite3.connect("votes.sqlite")
cursor = connection.cursor()
cursor.execute("INSERT INTO votes VALUES(?, ?)", (login, vote,))
connection.commit()
connection.close()
\end{lstlisting}
\caption{Example of~a~code using the~\emph{SQLite} library}
\end{figure}

\section{Vote Transfer}
This task is performed by~the~server for vote collection at~the~beginning of~the~counting period.

%The~vote transfer program selects encrypted votes from~the~database of~votes and saves all votes different from~the~special \texttt{0} vote in~\emph{JSON} encoded file $F$.
The~vote transfer program selects encrypted votes from~the~database of~votes to~produce a~list of~all votes different from~the~special \texttt{0} vote. Let us call this list $V$. The~list $V$ is, then, saved in~\emph{JSON} encoded file $F$. JavaScript Object Notation (JSON) is a~data~interchange format~derived from~the~JavaScript object literals defined in~the~\emph{ECMAScript Programming Language Standard}. It is text-based and language-independent. It can~represent \emph{null}, \emph{numbers}, \emph{boolean~values}, \emph{strings}, \emph{arrays} and \emph{objects}. Specification of~\emph{JSON} is contained in~\cite{JSON}. %The~encrypted votes are stored in~an~array of~strings, each element representing a~single vote.

The~file $F$ is saved to~a~USB flash drive by~the~voting commission and transferred to~the~machine for vote counting.

\section{Vote Counting}
The~\emph{JSON} file $F$ containing the~votes is moved from~the~USB device to~a~particular directory. Also, the~election private key is reconstructed and saved in~a~particular file. There, it is read by~the~counting program running on the~machine for vote counting. The~file undergoes several procedures in~order to~have the~votes extracted, decrypted and counted:
\begin{enumerate}
\item The~original list of~votes $V$ is reconstructed from~$F$. Each vote $v \in~V$ is a~string representing the~vote in~the~desired format~of~the~vote.
\item Each vote $v \in~V$ is decrypted and parsed. The~result of~parsing is a~list of~characters $C$ which form the~string $v$.
\item Each vote, now represented by~$C$, is checked whether it is of~the~proper format~and answers for each candidate, represented by~the~integer values, are extracted.
\item All valid votes with extracted values are counted and the~final result is published by~the~machine.
\end{enumerate}

Vote validation, extraction and counting are more deeply described in~the~following subsections.

\subsection{Validation}
The~purpose of~the~validation is to~check whether the~vote is of~the~proper format. More formally: \newline Let the~vote $v = v_1v_2....v_n$, where $n \in~\mathbb{N}$ and $v_1, ..., v_n \in~\{\texttt{0},...,\texttt{9}$, \texttt{<}, \texttt{>}, \texttt{\#}$\}$. It can~be noticed that~subsequences that~contain~\texttt{0} immediately after~\texttt{<} or \texttt{\#}, for example \texttt{<1\#09>}, are permitted. Such case can~be represented in~a~way that~the~initial \texttt{0} be omitted from~the~final representation of~the~value during~the~extraction process. According to~this, \texttt{09} would become $9$.

We call the~vote $v$ \emph{valid} if it is accepted by~deterministic finite automaton $A~=\{Q, \Sigma, \delta, q_0, F\}$, 
where $Q = \{q_0, ..., q_4\}$, $\Sigma~= \{\texttt{0},...,\texttt{9}$, \texttt{<}, \texttt{>}, \texttt{\#}$\}$ and $F = \{q_0\}$. We do not describe the~$\delta$ function, since it can~be easily reconstructed from~the~diagram in~Figure \ref{fig:automaton}. There, the~word \emph{digit} represents any symbol of~digit \texttt{0},...,\texttt{9}.

\begin{figure}[h]
\begin{center}
\label{fig:automaton}
\begin{tikzpicture}[shorten >=1pt,node distance=3cm,on grid,auto] 
   \node[state,initial,accepting] (q_0)   {$q_0$}; 
   \node[state] (q_1) [above right=of q_0] {$q_1$}; 
   \node[state] (q_2) [below right=of q_1] {$q_2$}; 
   \node[state](q_3) [below right=of q_2] {$q_3$};
   \node[state](q_4) [below right=of q_0] {$q_4$};
    \path[->] 
    (q_0) edge  node {\texttt{<}} (q_1)
    (q_1) edge  node  {\emph{digit}} (q_2)
    (q_2) edge  node {\texttt{\#}} (q_3) 
          edge [loop right] node {\emph{digit}} ()
    (q_3) edge node {\emph{digit}} (q_4)
    (q_4) edge  node {\texttt{>}} (q_0) 
          edge [loop left] node {\emph{digit}} ();
\end{tikzpicture}
\end{center}
\caption{Deterministic finite automaton accepting valid votes}
\end{figure}

The~reason why we describe $A$ is that~the~code for vote validation is an~actual simulation of~this automaton. Therefore, we find $A$ a~good abstraction of~the~vote validation process as it is implemented in~our voting system.

%We call the~vote $v$ \emph{valid} when $v_1=$\texttt{<} and for each $v_i,v_{i+1} \in~v$, $i \in~\mathbb{N}$:
%\text{\texttt{1},...,\texttt{9}}\}       & \quad \text{if } v_i \in~\{\text{\texttt{<}, \texttt{\#}}


\subsection{Extraction}
One of~the~most important task of~this part of~our system is to~extract all the~ordered pairs \emph{(canditate\_number, value)} from~all valid votes. During~this process, it is also double-checked whether the~individual votes do not contain~more positive answers than~allowed.

At~the~beginning, a~counter $q$ for the~\emph{positive} answers is initialised and set to~zero. Dictionary $D$ is initialised. In~$D$, candidate numbers $c$ would form the~key for which a~value $v$ associated with the~answer is stored. Then, list $C$, representing the~vote in~the~valid format, is looped through. Let us show to~the~reader how a~single dictionary element $D[c]$ with value $v$ is formed.
 
It can~be said that~the~algorithm works in~two distinct states. The~first state is when a~\texttt{<} symbol has been read, but the~next \texttt{\#} symbol has not been reached, yet. In~this state, a~candidate number is read digit by~digit. Let $k$ be the~already read part of~the~candidate number $c$ and the~first $n$ digits thereof~have already been read. We assume that~there is at~least one digit before the~next \texttt{\#} symbol. We move on to~the~next digit $d$. Because the~numbers are represented in~the~standard decimal position notation, $k$ is changed in~the~following manner: $k:= 10k + d$. This process is finished when we reach a~\texttt{\#} symbol. Finally, we do $c:=k$.

In~the~second state we have read a~\texttt{\#} symbol, but the~next \texttt{>} symbol has not been read, yet. In~this state we form the~value $v$ associated with the~candidate number similarly to~what~was done in~the~first stage. When we reach a~\texttt{>} symbol, we assign $v$ its value and we do  $D[c]:=v$. If $v$ equals the value that represents the~\emph{positive} answer, we increment $q$ by $1$.

When the~whole vote was processed, we check whether $q$ has not exceeded the~maximum number of~positive answers defined by~the~election commission. If so, we return zero, otherwise we return $D$.

During extraction, we deal with an~attempt to~send an~invalid vote that contains two keys $k_i$ and $k_j$ such that $k_i = k_j$, where $i < j$. According to~the~process, value $v_j$ associated with~$k_j$ rewrites value $v_i$ associated with~$k_i$.

\begin{figure}[h]
\begin{lstlisting}[language=Python]
num = 0  # checks the number of yeses
values = {}
j = 1  # the first number in the string
while j < len(l):
    key = 0
    val = 0
    while l[j] != '#':
        key = 10*key + int(l[j])
        j += 1
    j += 1  # the first number after "#"
    while l[j] != '>':
        val = 10*val + int(l[j])
        j += 1
    if val == 1:
        num += 1
    values[key] = val
    j += 2  # the first number after "<"

if num > CAND_NUM:
    return 0

return values
\end{lstlisting}
 \caption{Code for extraction of~the~values}
 \end{figure}   

\subsection{Counting}
This is the~final part of~our election system. The~votes with extracted values are counted and the~final result is displayed in~this part.

Let us recapitulate that~we have $3$ types of~possible answers: \emph{positive}, \emph{negative} and \emph{neutral}. At~the~beginning, $3n \in~\mathbb{N}$ variables are initialised and set to~zero, where $n$ is the~number of~candidates. Let $i$ be a~particular candidate number and $j$ be a~value representing a~particular type of~answer. Then, $c[i][j]$ represents count of~answers of~type $j$ given to candidate number $i$. %Each represents the~count of~particular type of~answers given to~a~particular candidate. 
For every candidate $i$, it is looped through every vote $v$. For every value $j$, if $j$ is associated with key $i$ in vote $v$, then $c[i][j]$ is incremented by $1$.
%and the~initialised variables are incrementing accordingly.

Finally, it is displayed for every candidate how many positive, negative and neutral answers they were given.

During the counting process, we also deal with~votes that contain invalid candidate numbers or invalid voting options. These are not counted because the~process looks for~particular candidate numbers and particular voting options.

\section{Administration}
\label{sek:administration}
Some administration programs are also included in~our election system. They are used by~the~election commission for \emph{key generation and secret sharing of~the~election private key} and to~\emph{administer the~paper voting process}. They are mostly short, similar to~the~parts of~code already described, or they often use packages and libraries described at~the~beginning of~this chapter. We do not spend much time describing them, but we provide a~complete list of~them and shortly describe the~function of~each one:
\begin{itemize}
\item \textbf{Paper voting.} This is used by~the~election commission to~record that~a~voter has voted traditionally. The~election commission uses their internal \emph{UK login} and password to~log in, but they do not authenticate with \emph{Cosign}. Instead, they use a~special login~page. Then, a~form displays and they type in~the~\emph{UK login} of~the~voter that~is voting traditionally and confirm it.
\item \textbf{Key management.}{ Here, the~electronic election public and private keys can~be generated and the~private key can~be shared among members of~the~election commission. 

In~order to~generate the~election certificate and private key, the~election commission installs \emph{OpenSSL} to~a~special computer different from~the~server for vote collection. It might also be done on the~machine for vote counting, but we recommend not to~use it for security reasons. This command can~be executed to~generate the~election \emph{X.509} certificate and the~private key: \texttt{openssl req -newkey rsa:1024 -nodes -x509 -days 365 -out certificate.pem}. It creates the~certificate saved to~\texttt{certificate.pem} and the~private key saved to \texttt{privkey.pem} \cite{M2Crypto}. The~certificate is distributed in~a~way that~a~special \emph{JavaScript} file that~contains the~certificate is written and the~private key is shared among members of~the~election commission using a~secret-sharing program also attached to~this thesis. The~original file \texttt{privkey.pem} must be discarded.}

We have also implemented a~secret-sharing program. It splits the~given key into several parts using \texttt{secretsharing} library for \emph{Python}. The~shares are then given to~the~members of~the~election commission. Then, it reconstructs the~key from~shares given by~some of the~members.

We have not implemented any key generation program. However, we briefly describe such a~program. The~role of this program is to~generate the \emph{S/MIME} certificate and the~private key of the~election. In the~first place, it must not save the~generated private key to the~disk, but directly split it into~parts of a~shared secret. This is very important because the~private key must not be revealed. Then, the~program would share the~certificate among~the voters. This would help the~commission and decrease the~number of~operations they need to~do during~the~initialisation period.

%here a~requirement on~such a~program

%\item \textbf{Login~encryption.} We find it necessary to~encrypt the~voter's \emph{UK login} due to~security reasons. This service enables the~election commission to~generate \emph{RSA} public and private keys stored on the~server for vote collection.
\end{itemize}
%\section{Cosign Configuration}