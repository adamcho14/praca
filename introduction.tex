\chapter*{Introduction}
\addcontentsline{toc}{chapter}{Introduction} % rucne pridanie do obsahu
\markboth{Introduction}{Introduction} % vyriesenie hlaviciek
%This is the~introduction to~our bachelor thesis, introducing its important aspects.
Lives of~modern people are accompanied with modern digital technologies. Many people cannot even imagine their life without~a~computer devices. Governments and companies often promote their use in~a~variety of~fields. Since its beginning, Internet has spread out throughout the~whole society and services like Internet banking, social networks and e-mail are used on~daily basis by~plenty of~people in~this world. 

There are many attempts to~fully digitalise governmental administration, including the~election process. However, this raises several questions of~trust and many usability, security and accuracy requirements that every electronic voting system should follow have been defined.

Throughout the~centuries, several traditional voting systems have been developed. These include \emph{ostraca}—pieces of~broken pottery having been used in~Athens—or classic ballot papers in~envelopes, as we know them today. These systems have been being improved over a~long period of~time and they represent a~confidential way of~expressing one's opinion. 

Despite reliability of~traditional voting mechanisms, many institutions are interested in~introducing modern technologies to~the~field of~voting. These include either special-purpose machines or personal computers. Use of~these technologies, of~course, makes us meet new challenges. It took the~previous generations several centuries to~solve many usability, security and reliability issues makers of~electronic voting systems need to~solve almost immediately. Despite all the~considerations \cite{Rubin}, electronic—particularly Internet—voting is on~its best way to~become very popular.

Using cryptographic methods, such as blind signatures, anonymous channels or homomorphic encryption \cite{Haenni}, it is possible to~find an~elegant way to~solve all these issues. There have been quite a~few attempts to~design a~trustworthy election system. Some of~the~most successful approaches are election systems developed for~Estonia \cite{Maaten, Springall}, Norway \cite{Gjosteen2010, Gjosteen2012, Gjosteen2015} and Switzerland \cite{Swiss}. They have been run as pilot projects by~the~countries' governments.

Out faculty has also expressed their interest in~electronic voting. It is planned to~be used by~our students to~choose their representatives it our faculty's academic senate. In~this thesis, we develop and implement a~minimal, yet secure remote electronic election system. It can enable the~students to~cast a~vote with use of~nothing but an~Internet browser in~their personal computer or mobile device. We consider all the~limitations of~such technology that can have impact on~our solution. 

Finally, we analyse our solution and we try to~figure out to~what extent it meets defined usability, security and accuracy requirements.